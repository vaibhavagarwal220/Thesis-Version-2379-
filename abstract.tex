%\clearpage
%\thispagestyle{empty}
%~\clearpage
\begin{center}
{\Large \bf ABSTRACT}
\end{center}
\noindent

Deception is the process of making someone believe in information that is not entirely true. It may provide a promising realtime solution against cyber-attacks. In this thesis, we discuss a human-in-the-loop real-world simulation tool called HackIT, which could be configured to create different cyber-security scenarios involving deception. 

We describe how HackIT can be used to create networks of different sizes; use deception and configure different web-servers as honeypots; and, create fictitious ports, services; create any number of fake operating systems, and fake files on honeypots. Then, we discuss a case-study conducted using HackIT on two different conditions where adversaries were tasked with stealing information from a simulated network over multiple rounds. In the first condition, deception occurred early; and, in the other condition, it occurred late. Results revealed that participants used different attack strategies across the two conditions.

Next, We investigated the effects of several factors such as ratio of honeypots and network-type in different hacking scenarios. We discuss the Reconnaissance Deception System(RDS) used in different hacking scenarios. The RDS approach was compared with simple Non-RDS and combination of the former two approaches. Results revealed the effectiveness of RDS approach over the Non-RDS approach.

Finally, we discuss the potential of using HackIT in helping cyber-security teams understand adversarial cognition in the laboratory and using this information and using it in real networks to enhance the cyber security of these networks.\\
{\bf Keywords}: Defenders, Honeypots, Hackers, Attack, Cybersecurity, Simulation tools, Network topology

\newpage
~\clearpage 
